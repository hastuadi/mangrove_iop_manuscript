%% LyX 2.3.7 created this file.  For more info, see http://www.lyx.org/.
%% Do not edit unless you really know what you are doing.
\documentclass[a4paper]{jpconf}
\usepackage[latin9]{inputenc}
\usepackage{graphicx}

\makeatletter
%%%%%%%%%%%%%%%%%%%%%%%%%%%%%% Textclass specific LaTeX commands.
\newcommand{\lyxaddress}[1]{
	\par {\raggedright #1
	\vspace{1.4em}
	\noindent\par}
}

\makeatother

\usepackage{babel}
\begin{document}
\title{Satellite-based Mangrove species abundance estimate using Machine
Learning ensemble}
\author{Hastuadi Harsa, Gathot Winarso, Kuncoro Teguh Setiawan and Wikanti
Asriningrum}

\lyxaddress{\address{Remote Sensing Research Center - Research Organization for Aeronautics and Space\\
Indonesia National Research and Innovation Agency}}

\ead{hastuadi.harsa@brin.go.id}
\begin{abstract}
The mangrove ecosystem is a vital feature in a coastal area, playing
a critical role in carbon sequestration beneath the soil. Carbon preservation
capacity varies among different species of mangrove. Thus, by quantifying
the number of mangrove species in a given area, the volume of carbon
sequestered can be estimated. Satellite imagery is highly effective
for gathering such data across vast territories. In this study, we
present an evaluation of mangrove species abundance across a large
coastal area using Landsat satellite imagery. We employed machine
learning algorithms to classify species based on spectral field observation
data to achieve this. These algorithms were trained individually and
ensembled to enhance prediction performance. There are 466 models
generated in a two-hour training phase. After assessing these models,
we identified that a stacked ensemble consisting of Deep Learning,
two Distributed Random Forests, a Generalized Boosting Model, a Generalized
Linear Model, and Extreme Gradient Boosting algorithms has the most
superior predictive accuracy. The model achieved a mean accuracy value
of 95\% when tested on observation data. After applying the best model
to the satellite data, our results indicate that Rhizophora Apiculata
and Excoecaria Agallocha are the two most abundant mangrove species
in the study area, covering 17.71\% (19502.37 Ha) and 10.49\% (11549.79
Ha), respectively. 
\end{abstract}

\section{Introduction}

Carbon trading serves as a climate change mitigation strategy by creating
economic incentives for reducing greenhouse gas emissions \cite{Zheng2023}.
The approach enables industries and countries to trade carbon credits,
allowing those exceeding emission reduction targets to sell surplus
credits to entities struggling to meet their targets \cite{Xu2023,Ke2023}.
By valuing carbon emissions and facilitating their market-driven reduction,
carbon trading promotes sustainable practices and contributes to overall
emission reductions\cite{Cetera2022}.

Mangrove forests, characterized by their unique coastal ecosystems,
are known for their exceptional capacity to sequester and store carbon\cite{Sondak2019}.
These ecosystems, comprising mangrove trees and associated vegetation,
act as a vital carbon preservers by absorbing atmospheric carbon dioxide
(CO2) and retaining it within their biomass and sediments\cite{Wong2020}.
Mangrove plants possess the ability to sequester carbon at rates several
times higher than terrestrial forests, making them crucial allies
in climate change mitigation efforts\cite{Nasir2018}. 
\begin{figure}
\noindent \centering{}\includegraphics{images/studyarea2}\caption{\label{fig:studyarea}The study area, located in the western part
of Sumatra, Indonesia, was within the intersection of the available
satellite data scene and the geospatial mangrove vector area.}
\end{figure}

However, accurately estimating the carbon stock of mangrove forests
requires understanding the different carbon sequestration potentials
among various mangrove species\cite{Lassalle2022}. Different species
exhibit varying levels of biomass and different growth rates, directly
impacting their carbon storage capacity\cite{Wang2019,Navarro2019}.
Hence, identifying and quantifying mangrove coverage, accounting for
species-level variations, become essential for precise carbon stock
assessments\cite{Maeda2016,Zheng2022,Jiang2022}.

In recent years, remote sensing techniques, particularly satellite
imagery, have revolutionized ecological studies by providing a cost-effective
and efficient means of mapping large areas\cite{Ma2019,Atmaja2022}.
Satellite imagery enables the identification and monitoring of mangrove
forests at regional and global scales, facilitating comprehensive
assessments of carbon stock estimates\cite{Pham2019,Gandhi2019}.
Integrating satellite imagery with Machine Learning (ML) algorithms
allows for advanced analysis of spectral bands to predict mangrove
species, aiding in species-specific carbon stock calculations\cite{Hsu2020}.

This paper describes the use of ML for predicting mangrove species
from Landsat satellite imagery bands. Therefore, the carbon stock
estimate can be measured subsequently using the information from the
predicted mangrove species abundance.

\section{Data and methods}

The study area was located within the Sembilang National Park in Banyuasin
Regency, along the east coast of Sumatra, Indonesia. The area covered
by a red rectangle at the top-left of Figure \ref{fig:studyarea}shows
the study area. This study collected three distinct data types: the
mangrove area represented as polygons in a geospatial vector format
file, field-observation data, and Landsat satellite data. The Indonesia
Ministry of Environment and Forestry obtained the geospatial vector
data. The data determined the boundary of the mangrove area. The field-observation
data were collected within the mangrove area, employing a spectrometer
instrument. The instrument displayed the spectral characteristics
of mangrove leaves at selected sampling locations. Each observed mangrove
leaf was analyzed across seven distinct spectral bands to unveil a
comprehensive understanding of its unique attributes. The measurement
values the spectrometer gave, and the observed mangrove leaf species
were then recorded. Eventually, the observation data contained eight
variables, i.e., seven bands of spectrums and their related species.
In the measurement process, only the broadest leaf in the neighborhood
was selected as the sample data, and there were 21 mangrove species
identified in the study area.

The satellite data were obtained from a single Landsat Satellite imagery
scene and covered some geospatial vector data. The data were represented
in a geospatial raster format. The raster had a homogeneous spatial
resolution of approximately 30 meters. The superposition of satellite
data and the geospatial vector data are displayed at the bottom-left
of Figure \ref{fig:studyarea}. The satellite data also contained
seven bands, as in the field observation data. Both observation and
satellite data bands were referred to as B1 to B7, as the abbreviation
of a band and its corresponding satellite sensor. The distinction
between the observed and satellite data was that the observed data
contained eight variables, while the satellite data contained only
seven. The eighth variable of the observation data was labeled as
'class', denoting the species of each observed mangrove leaf. A composition
of Red-Green-Blue (RGB) bands of satellite raster data is displayed
on the right side of Figure \ref{fig:studyarea}. This image illustrates
a true color composition as a visible color spectrum. The RGB values
were denoted by B1, B2, and B3 for Blue, Green, and Red colors, respectively.
The red polygon reveals the overlapping area between the mangrove
geospatial vector and satellite raster data.

The main effort in estimating the mangrove species' abundance was
presenting an absent variable to the satellite data, which was present
in the observation data. The absent variable was the eighth variable,
i.e., class. Since the class variable was the dependent variable of
each datum in the observation data, it was essential to extract the
underlying interaction optimally among all band variables as the independent
variables. The satellite bands interaction characterized the class
variable. In ML terms, this process is known as classification\cite{Boehmke2020,Molnar2023,Scott2023}.
This study employed some ML algorithms \cite{LeDell2020}to build
models that extract the underlying relationship between the independent
and the dependent variables of observation data.

Before building the model, all values of the independent variables
in the observation data were standardized to have zero mean and a
standard deviation of one. By standardizing the independent variables,
all independent variables would have the same amount of contribution
to the model building. In addition, the standardization procedure
would exaggerate the interaction pattern among variables. The standardized
observation data were then duplicated into 20 folds to provide a cross-validation
element. Each fold was separated into two components: training and
validation. The ratio of training and validation for all folds was
80\% and 20\%, respectively. The order of training and validation
data were chosen randomly within each fold.

The standardized observation data were then fed to the ML algorithms
as training data. The algorithms chosen in this study were Distributed
Random Forest (DRF)\cite{Geurts_2006}, Generalized Linear Model (GLM)\cite{Dunn_2005},
Extreme Gradient Boosting (XGBoost)\cite{Mitchell2017,Chen2016},
Generalized Boosting Machine (GBM)\cite{Click2015,Malohlava2023},
and Deep Learning (DL)\cite{Candel2015,Ghorbani2020,Elsayad2020}.
This ML algorithm each had different model parameters. All algorithms
were trained multiple times using various configurations following
the needs of each algorithm's parameter. Each algorithm setup produced
a single model. The models were stored for further performance comparison.
A total of two-hour training sessions was designated as the maximum
training time. Models with good performance were also ensembled to
outperform each algorithm's best model. 
\begin{figure}
\noindent \centering{}\label{fig:performance}\includegraphics[width=0.75\textwidth]{images/akurasi_dan_fold}\caption{Model performance: (a) the top-10 best models out of 466, sorted by
their accuracy, (b) best model accuracy in each fold.}
\end{figure}

One final best model was then chosen from the model's collection.
This model was adopted to predict mangrove species at pixels of the
satellite bands data. The satellite bands data were also standardized
first as the training data. The prediction was performed only on the
pixels located inside the mangrove polygons. All predicted species
were then tabulated to estimate the abundance of species within the
mangrove polygons.

\section{Results and discussion}

The modeling of the spectral combination value in composing the mangrove
species yielded 466 models. Accuracy metrics determined these models'
performance. A single accuracy value was defined as comparing the
number of correct predicted species and the number of data. Since
there were 20 folds of cross-validation datasets, the accuracy representing
a model was determined as the mean of all accuracy of the 20 folds
cross-validation dataset. Furthermore, the mean accuracy was taken
as the models' sorting variable. After being sorted, the best model
was a stacked ensemble model composed of six base models, i.e., DL,
two DRFs, GBM, GLM, and XGBoost. The meta-learner algorithm for ensembling
these base models was a GLM algorithm. The mean accuracy of the best
model was 95.04\%, as shown in Figure \ref{fig:performance} (a),
together with other sorted top-nine models. This accuracy value is
qualitatively good. The individual accuracy of the best model for
each fold is displayed in Figure \ref{fig:performance} (b). There
are two groups of accuracy values in each fold presented in Figure
\ref{fig:performance} (b), i.e., the accuracy for all species in
each fold data set and the mean accuracy for each class in each fold.
The accuracy for all species in each cross-validation fold ranged
between 86\% to 100\% and 90\% to 100\% per species.

The spatial distribution of species classification identified by the
best models in the satellite data is shown in Figure \ref{fig:prediction}.
Most species inhabited a clustered region. There are two important
variables related to species identification, i.e., the number and
the spread. The greater number of a species would result in an easier
identification. On the other hand, the wider the spread of a species
would result in a harder identification. If a species scattered over
the area, then it must present in a great number for an easier identification.
The effect of species spread is shown in Figure 4. The figure emphasizes
the location of the five most abundant species for a better location-related
analysis. Rhizophora Apiculata, Excoecaria Agallocha, and Sonneratia
ovata, the top three most abundant species, are mostly concentrated
at a specific location. At the same time, Avicennia Officinalis and
Ceriops Tagal are more distributed over the area. A complete list
of all species counts is displayed in Table \ref{table:speciescoverage}.
\begin{figure}
\noindent \centering{}\label{fig:prediction}\includegraphics[width=1\textwidth]{images/prediction2-01}\caption{The best model classification output.}
\end{figure}

In Table \ref{table:speciescoverage}, each pixel of satellite data
were tallied by its classification predicted by the model. Rhizophora
Apiculata and Excoecaria Agallocha occupied 216693 and 128331 pixels,
respectively, or 17.71\% and 10.49\% in the percentage of the total
pixels. The least species is Acrostichum sp, covering 1667 pixels
or 0.14\% of the total pixels. Since the resolution of Landsat is
30$\times$30 $m^{2}$, therefore the coverage area in Hectare (Ha)
can be obtained by multiplying the number of pixels with 0.09, resulting
in the area coverage of 19502.37 Ha for Rhizophora Apiculata as the
most abundant species and 11549.79 Ha for Excoecaria Agallocha as
the second most abundant species. A complete list of all species coverage
is presented in Table \ref{table:speciescoverage}.
\begin{figure}
\noindent \begin{centering}
\label{top5}\includegraphics[width=1\textwidth]{images/distribusi_kelas_terbanyak}
\par\end{centering}
\caption{Distribution of five most-abundance mangrove species identified by
the best model.}

\end{figure}

{
\small
\begin{table} 
\caption{\label{table:speciescoverage}Area coverage} 
\begin{center} 
\begin{tabular}{crr@{\extracolsep{0pt}.}lr@{\extracolsep{0pt}.}l} 
\br 
Species & Pixel count & \multicolumn{2}{c}{Percentage } & \multicolumn{2}{c}{Coverage (Hectare) }
\tabularnewline 
\hline 
Rhizophora Apiculata & 216693 & 17&71 & 19502&37\tabularnewline 
Excoecaria agallocha & 128331 & 10&49 & 11549&79\tabularnewline 
Sonneratia ovata & 121761 & 9&95 & 10958&49\tabularnewline 
Avicennia officinalis & 101432 & 8&29 & 9128&88\tabularnewline 
Ceriops tagal & 96629 & 7&90 & 8696&61\tabularnewline 
Avicennia alba & 86897 & 7&10 & 7820&73\tabularnewline 
Derris trifoliata & 71376 & 5&83 & 6423&84\tabularnewline 
Bruguiera cylindrica & 68186 & 5&57 & 6136&74 \tabularnewline 
Rhizophora mucronata & 64220 & 5&25 & 5779&80 \tabularnewline 
Xylocarpus granatum & 53867 & 4&40 & 4848&03 \tabularnewline 
Bruguiera parviflora & 51098 & 4&18 & 4598&82 \tabularnewline 
Nypa fruticans & 36129 & 2&95 & 3251&61 \tabularnewline 
Sonneratia alba & 29928 & 2&45 & 2693&52 \tabularnewline 
Acrostichum speciosum & 29282 & 2&39 & 2635&38 \tabularnewline 
Avicennia marina & 18256 & 1&49 & 1643&04 \tabularnewline 
Bruguiera sexangula & 16083 & 1&31 & 1447&47 \tabularnewline 
Acrostichum aureum & 10411 & 0&85 & 936&99 \tabularnewline 
Acanthus ilicifolius & 9181 & 0&75 & 826&29 \tabularnewline 
Bruguiera gymnorrhiza & 6378 & 0&52 & 574&02 \tabularnewline 
Rhizophora stylosa & 5680 & 0&46 & 511&20 \tabularnewline 
Acrostichum sp & 1667 & 0&14 & 150&03\tabularnewline 
\br 
\end{tabular} 
\end{center} 
\end{table}
}

\section{Conclusion}

An estimate method of mangrove species abundance has been presented
in this paper. The method applied some ML Machine Learning (ML) algorithms
to provide species classification models. The algorithms used mangrove
species spectrum data from field observation as their learning materials.
Some best ML models were then ensembled to develop a better model.
The best model was then implemented as a classifier of Landsat satellite
data. Based on the classification output, the coverage of each mangrove
species in the study area can be estimated. Rhizophora Apiculata and
Excoecaria agallocha are the two most abundant mangrove species found.
By quantifying the abundance of mangrove species, the amount of carbon
sequestered can also be computed as a consideration element in carbon
trading. 

\section*{References}

\bibliographystyle{iopart-num}
\addcontentsline{toc}{section}{\refname}\bibliography{all2}

\end{document}
