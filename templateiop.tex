\documentclass[a4paper]{jpconf}
\usepackage{graphicx}
\begin{document}
\title{$title$}

\author{$author$}

\address{$alamat$}

\ead{hastuadi.harsa@brin.go.id}

\begin{abstract}
$abstrak$
\end{abstract}

$body$

\section*{References}
\begin{thebibliography}{9}

\bibitem{iopartnum} IOP Publishing is to grateful Mark A Caprio, Center for Theoretical Physics, Yale University, for permission to include the {\tt iopart-num} \BibTeX package (version 2.0, December 21, 2006) with  this documentation. Updates and new releases of {\tt iopart-num} can be found on \verb"www.ctan.org" (CTAN). 

\bibitem{Zheng2023} Zheng Y, Tan R and Zhang B 2023 The joint impact of the carbon market on carbon emissions, energy mix, and copollutants \emph{Environmental Research Letters} \textbf{18(4)} 045007 https://doi.org/10.1088/1748-9326/acca98

\bibitem{Cetera2022} Cetera K 2022 Recognition of forest carbon rights in Indonesia: a constitutional approach \emph{Lentera Hukum} \textbf{9(1)} 151-176 https://doi.org/10.19184/ejlh.v9i1.29331

\bibitem{Xu2023} Xu S, Pan W and Wen D 2023 Do carbon emission trading schemes promote the green transition of enterprises? Evidence from China \emph{Sustainability} \textbf{15} 6333 https://doi.org/10.3390/su15086333

\bibitem{Ke2023} Ke S, Zhang Z and Wang Y 2023 China's forest carbon sinks and mitigation potential from carbon sequestration trading perspective \emph{Ecological Indicators} \textbf{148} 110054 https://doi.org/10.1016/j.ecolind.2023.110054

\bibitem{Sondak2019} Sondak C F A, Kaligis E Y, Robert A B 2019 Economic valuation of Lansa mangrove forest, North Sulawesi, Indonesia \emph{Biodiversitas} \textbf{20(4)} 978-986 https://doi.org/10.13057/biodiv/d200407

\bibitem{Wong2020} Wong C J, James D, Besar N A, Kamlun K U, Tangah J, Tsuyuki S, Phua M 2020 Estimating mangrove above-ground biomass loss due to deforestation in Malaysian Northern Borneo between 2000 and 2015 using SRTM and Landsat
images \emph{Forests} \textbf{11(1018)} 1018 https://doi.org/10.3390/f11091018

\bibitem{Nasir2018} Nasir S, Muhammad H, Novi S A 2018 Modeling mangrove 'blue carbon' ecosystem service in Jakarta bayasan impact of coastal development \emph{E3S Web of Conferences} \textbf{73} 04023 https://doi.org/10.1051/e3sconf/20187304023

\bibitem{Lassalle2022} Lassalle G, Filho C R deS 2022 Tracking canopy gaps in mangroves remotely using deep learning \emph{Remote Sensing in Ecology and Conservation} \textbf{8(6)} 890-903 https://doi.org/10.1002/rse2.289>

\bibitem{Xin2023} Xin W, Kai L, Jingjing C, Yuanhui Z, Ziyu W 2023 Estimation of mangrove aboveground biomass in China using forest canopy height through an allometric equation \emph{Redai Dili} \textbf{43(1)} 1-11 https://doi.org/10.13284/j.cnki.rddl.003616

\bibitem{Wang2019} Wang D, Wan B, Qiu P, Zuo Z, Wang R, Wu X 2019 Mapping height and aboveground biomass of mangrove forests on Hainan island using UAV-LiDAR sampling \emph{Remote Sensing} \textbf{11(18)} 2156 https://doi.org/10.3390/rs11182156

\bibitem{Navarro2019} Navarro J A, Algeet N, Fern??ndez-Landa A, Esteban J, Rodr??guez-Noriega P, Guill??n-Climent M L 2019 Integration of UAV, Sentinel-1, and Sentinel-2 data for mangrove plantation aboveground biomass monitoring in Senegal \emph{Remote Sensing} \textbf{11(1)} 77 https://doi.org/10.3390/rs11010077

\bibitem{Maeda2016} Maeda Y, Fukushima A, Imai Y, Tanahashi Y, Nakama E, Ohta S, Kawazoe K, Akune N 2016 Estimating  carbon stock changes of Mangrove forests using satellite imagery and airborne lidar data in the South Sumatra state,
Indonesia The International Archives of the Photogrammetry, \emph{Remote Sensing and Spatial Information Sciences} \textbf{XLI-B8} 705-709 https://doi.org/10.5194/isprs-archives-XLI-B8-705-2016

\bibitem{Zheng2022} Zheng Y, Takeuchi W 2022 Estimating mangrove forest gross primary production by quantifying environmental stressors in the coastal area \emph{Scientific Reports} \textbf{12(1)} 1-14 https://doi.org/10.1038/s41598-022-06231-6

\bibitem{Jiang2022} Jiang X, Zhen J, Miao J, Zhao D, Shen Z, Jiang J, Gao C, Wu G, Wang J 2022 Newly-developed three-band hyperspectral vegetation index for estimating leaf relative chlorophyll content of mangrove under different severities of pest and disease \emph{Ecological Indicators} \textbf{140} 108978 https://doi.org/10.1016/j.ecolind.2022.108978

\bibitem{Ma2019} Ma C, Ai B, Zhao J, Xu X, Huang W 2019 Change detection of mangrove forests in coastal Guangdong during the past three decades based on remote sensing data \emph{Remote Sensing} \textbf{11(8)} 921https://doi.org/10.3390/rs11080921

\bibitem{Atmaja2022} Atmaja T, Fukushi K, Fukushi K 2022 Empowering geo-based AI algorithm to aid coastal flood risk analysis: a review and framework development \emph{ISPRS Annals of the Photogrammetry, Remote Sensing and Spatial Information Sciences} \textbf{3} 517-523 https://doi.org/10.5194/isprs-annals-V-3-2022-517-2022

\bibitem{Pham2019} Pham T, Yokoya N, Bui D, Yoshino K, Friess D 2019 Remote sensing approaches for monitoring mangrove species, structure, and biomass: opportunities and challenges \emph{Remote Sensing} \textbf{11(3)} 230 https://doi.org/10.3390/rs11030230

\bibitem{Gandhi2019} Gandhi S, Jones T 2019 Identifying mangrove deforestation hotspots in South Asia, Southeast Asia and
Asia-Pacific \emph{Remote Sensing} \textbf{11(6)} 728 https://doi.org/10.3390/rs11060728

\bibitem{Wicaksono2017} Wicaksono P 2017 Mangrove above-ground carbon stock mapping of multi-resolution passive remote-sensing systems \emph{International Journal of Remote Sensing} \textbf{38(6)} 1551-1578 https://doi.org/10.1080/01431161.2017.1283072

\bibitem{Hsu2020} Hsu A J, Kumagai J, Favoretto F, Dorian J, Martinez B G, Aburto-Oropeza O 2020 Driven by drones: improving mangrove extent maps using high-resolution remote sensing \emph{Remote Sensing} \textbf{12(23)} 3986 https://doi.org/10.3390/rs12233986

\end{thebibliography}

\end{document}


